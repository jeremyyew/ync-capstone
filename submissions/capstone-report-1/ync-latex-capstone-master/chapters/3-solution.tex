% Chapter Template

\chapter{Solution} % Main chapter title

\label{solution} % Change X to a consecutive number; for referencing this chapter elsewhere, use \ref{ChapterX}

%----------------------------------------------------------------------------------------
%	SECTION 1
%----------------------------------------------------------------------------------------


\section{A grammar of grammars}
The solution to these issues of abstract and concrete syntax is to develop a system that enforces explicit tactic application within a subset of Coq, amongst other rules. This system will be in the form of a program that takes as input a student's Coq file, as well as a grammar specification provided by the Lecturer. It will output warnings about instances where a syntax rule has been violated. (From here, I refer to 'the program' interchangeably as 'the tool' or 'the parser'.)

The program will be writen as an extension of the Emacs editor, so students can execute a command within Emacs to parse the current file, and thus use the tool interactively as they construct their proofs. 

The program will thus act as a set of safety rails for students to develop the right habits, in the spirit of learning for the first half of the course. As a result, students will have earlier, automated intervention on syntactical issues in their assignments, and the Lecturer can spend more time on substantive rather than superficial feedback. 

However, the parser should not 'hard-code' a grammar that only enforces particular syntax rules. Instead, it should accept and parse a grammar specification that is readable and easily editable, and enforce that grammar. In other words, we need to implement a grammar of grammars. This will allow the Lecturer to modify existing rules or extend them, without having to modify the source code of the parser. This will also make it easier for other course instructors or developers to modify the parser behaviour for their own needs. 

\section{Implementation trade-offs}
The first trade-off is whether to write a custom syntax parser by hand, or to use a parser generator.

A parser generator is a program that accepts a grammar specification as input, and automatically generates a parser that implements the grammar. I have identified parser generators intended to be run within Emacs, and written in Emacs Lisp (more on that below).

A parser generator would be ideal - we do not want to reinvent the wheel. Firstly, in theory, there is minimal to no programming to be done. Instead, we declare a grammar using some notation. 

Secondly, relying only on the declared grammar makes the program more extensible; it is easier for the Lecturer or other developers to modify existing rules or add new rules, since they do not need to touch underlying code.

However, it will be necessary to grasp the grammar that is required as input in the first place. 

A second trade-off relates to the case where we write a custom parser by hand: whether to write it using Emacs Lisp, or another language that I am more familiar with.

Of course, using a familiar language might mean that we get to a first working version faster. But an Emacs Lisp implementation has many benefits: 
\begin{itemize}
    \item Easier for other Emacs developers to build on  
    \item Integrated with Emacs editor features
    \item Run within Emacs. No need to call external process, no external dependencies, no interoperability issues (e.g. making a call to create a new thread, etc).  
\end{itemize}


This is why it is ideal to use a parser generator that generates Emacs Lisp code.  

\section{Current progress}
I have made progress in the following areas: 

\begin{itemize}
    \item Defining a subset of Coq grammar.
    \item Exploring different types of parsers for different types of languages.
    \item How to write a parser.
    \item How to write Emacs Lisp programs.
    \item How to write an Emacs extension.
\end{itemize}


In particular, I have written a script in Emacs Lisp that registers an Emacs interactive command, which may be executed while editing a Coq proof. The command is able to take the current buffer and use the Coq shell to parse it for Coq syntax errors. This command would be eventually used to run our custom parser; the idea is to first check that the input is syntactically correct with respect to Coq's grammar. This allows us to develop our custom parser assuming that the input code is already syntactically correct, hence we may define or provide a grammar that only encompasses a subset of Coq's grammar.

Additionally, I have also started trying to use the Semantic Bovine parser generator. Semantic is a framework for writing Emacs packages, and Bovine is a built-in parser generator provided by Semantic. It accepts a BNF (Backus-Naur Formation)-like grammar specification, and generates a parser in Emacs Lisp, which is ideal. The main obstacle for now would be the slightly obscure documentation. The first step I am aiming for would be to specify a simple grammar (such as for arithmetic expressions with infix notation), and ensure that it raises an appropriate error when given a syntactically incorrect input. 

\section{Challenges ahead}
The challenges ahead would be to decide on and proceed with an implementation of a program that can enforce a grammar addressing the two issues mentioned. Following which, I can then explore other rules mentioned. I will also be iterating on the tool based on usability feedback from both the Lecturer and FPP students, and the feedback will also be the measure of success for my project.