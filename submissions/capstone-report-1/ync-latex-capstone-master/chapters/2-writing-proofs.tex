% Chapter Template

\chapter{Writing proofs} % Main chapter title

\label{writing-proofs} % Change X to a consecutive number; for referencing this chapter elsewhere, use \ref{ChapterX}

%----------------------------------------------------------------------------------------
%	SECTION 1
%----------------------------------------------------------------------------------------

\section{What could go wrong?}
With programming languages, there are usually many ways to write the same program. In the same way, there are many equivalent representations of a Coq proof, because Coq is flexible and allows you to take shortcuts. However, for new learners, this flexibility can be counterproductive. In the context of FPP, several issues arise. 

\subsection{Abuse of tactics}
First, students may abuse tactics that have not been introduced in the course. 

When students get stuck on a proof, they might Google for related solutions or search the Coq documentation for anything that will 'solve' the proof. They might end up using a magical tactic, for example `trivial`, as in the latter version of the example proof below. 
\begin{verbatim}
Lemma SSSn_is_3_plus_n :
  forall n : nat,
  S (S (S n)) = 3 + n.
Proof.
  intro n.
  rewrite <- (Nat.add_1_l n).
  rewrite <- (plus_Sn_m 1 n).
  rewrite <- (plus_Sn_m 2 n).
  reflexivity.
Qed.
\end{verbatim}

\begin{verbatim}
Lemma SSSn_is_3_plus_n :
  forall n : nat,
  S (S (S n)) = 3 + n.
Proof.
  trivial.
Qed.
\end{verbatim}

Under the hood, the `trivial` tactic uses some heuristics to automatically try various strategies to solve the current formula. However, in the first half of the course the focus is for students to understand every single proof step they write, because if students cannot explain what they are doing, they do not really understand it. Therefore, using a tactic like `trivial` completely goes against the objective of the exercise. 

Yet these tactics still appear in student submissions, because they might still have the bad programmer mindset of "if it works, its fine". This causes time between resubmissions to be wasted on superficial feedback. 

\subsection{ Misuse of tactics }
Second, even when students use tactics that have been introduced, they may misuse them by taking shortcuts.

For instance, the rewrite tactic is used to apply a rule to the current formula. A rewrite rule is a function that expects specific terms in the formula as arguments; Coq will rewrite the given terms. For example, the rewrite rule below accepts three arguments, n, m, p.

\begin{verbatim}
Check Nat.add_assoc.
# Nat.add_assoc : forall n m p : nat, n + (m + p) = n + m + p. 
\end{verbatim}

However, Coq is flexible with the number of arguments you give it. As the example proofs below demonstrate, you could give the rewrite rule three, two, one or zero of the rewrite arguments required, and Coq will simply pick the first terms in the formula that it can apply the rule to. 
\begin{verbatim}
Proposition add_assoc_nested :
  forall a b c d e: nat,
    a + b + c + d + e = 
    a + (b + (c + (d + e))).
Proof.
  intros a b c d e.
  rewrite -> (Nat.add_assoc a b (c + (d + e)) ).
  rewrite -> (Nat.add_assoc (a + b) c (d + e) ).
  rewrite -> (Nat.add_assoc (a + b + c) d e ).
  reflexivity.
Qed.
\end{verbatim}
\begin{verbatim}
Proposition add_assoc_nested :
  forall a b c d e: nat,
    a + b + c + d + e = 
    a + (b + (c + (d + e))).
Proof.
  intros a b c d e.
  rewrite -> (Nat.add_assoc a b )
  rewrite -> (Nat.add_assoc (a + b) ).
  rewrite -> Nat.add_assoc.
  reflexivity.
Qed.
\end{verbatim}
However, in the first half of the course, the focus is on understanding the proof at a low level. Clearly, students need to be aware of exactly which terms they have changed at every step. Otherwise, they may get stuck in a proof because they applied a rewrite rule to the wrong term, or they might reach a solution without knowing how. Therefore, taking advantage of this shortcut goes against the spirit of the exercise. 

Furthermore, this issue is not easy to check manually, especially with assignments that are hundreds of lines long.

These two issues - abuse and misuse of tactics - correspond to issues of **abstract syntax** (what language constructs are represented in the grammar) and **concrete syntax** (what structures are used to represent language constructs) respectively. 

Therefore, it would be nice to have a system that can anticipate and identify both abstract and concrete syntax issues, to save both students and the Lecturer's time and help achieve the learning goals of the course.   


\subsection{Other issues}
Other issues I have discussed with the Lecturer include: 
\begin{itemize}
  \item arbitrary indentation levels for proof subcases 
  \item inconsistent naming
  \item breaking style conventions. 
\end{itemize}


All these issues seem to persist across the progression of the module, as well as iterations of the module, despite the Lecturer explaining to the students the rationale for following provided syntactical guidelines, and repeated reminders.

The idea is for the proposed tool to cut down on the amount of **'superficial'** feedback - e.g., 'don't use this tactic, because...', or 'this is bad style, please correct it in this way', etc. - that the Lecturer must give repeatedly to individual students, and instead automatically lead students towards solutions that only require **substantive** feedback - e.g., ideas to pursue, possible restructuring of the proof, etc. The less superficial feedback is required, the more time the Professor can spend on providing substantive feedback. Also, students will spend less effort correcting style errors if they do so immediately.

Yet, superficial feedback is not merely incidental. Superficial feedback reflects the formal concerns of the course and helps reinforces good programming habits, which will not only assist the learning experience of students, but benefit them in future endeavors. Therefore, the tool does not simply emphasize pedantic concerns; it makes concrete the formal training prescriptions of the course. 