% Chapter Template

\chapter{Context} % Main chapter title

\label{context} % Change X to a consecutive number; for referencing this chapter elsewhere, use \ref{ChapterX}

%----------------------------------------------------------------------------------------
%	SECTION 1
%----------------------------------------------------------------------------------------

\section{Introduction}
The goal of this project is to provide learning support for students enrolled in YSC3216: Functional Programming and Proving (FPP), by providing a tool that generates corrective suggestions for syntax issues, to be used by students to check their code.

FPP is a course in Yale-NUS College under the Mathematical, Computational and Statistical Sciences major, most recently offered in AY19/20 Semester 1 and taught by Professor Olivier Danvy. FPP introduces students to the Coq proof assistant, which is a system for writing and verifying formal proofs. Throughout the course, students learn that precision and orderliness in their code both reflects and encourages clarity of thought. Therefore, one of the primary learning goals for the first half of the course is to build muscle memory for basic proof techniques and programming habits.

To this end, I implement a program called the \code{proof-reader} that acts a set of ’safety rails’ to guide students towards good proving and programming habits via automated intervention on syntax issues. In particular, this tool will help enforce prescribed techniques: using only tactics introduced in the course, and applying tactics explicitly. Since these issues are often highlighted in written feedback from the instructor, the tool also supports students' learning by greatly reducing the feedback cycle and allowing the instructor to focus on substantive rather than superficial feedback.

The program relies on a grammar specification of a subset of Coq as an input to the tool, which the instructor may modify as the course evolves. The program's interface is a simple Emacs command, and is intended to be used interactively by students, both during proof editing and before submission.

\section{Functional programming (FP)}
Functional programming is a programming paradigm that models programs as mathematical functions. Students taking FPP are expected to have completed the Introduction to Computer Science module, which will have trained them in functional programming (amongst other concepts) using the language OCaml. Coq has a language of programs that is very similar to OCaml, and is in fact written in OCaml.

\subsection{Proving}
In mathematics, a proposition is a statement that either holds or does not hold; a proposition is also sometimes called a theorem or lemma. Proofs may be defined as a logical argument about whether a proposition holds. Proofs use logical rules to demonstrate that what we know or assume to be true – an axiom – implies the truth of something that we do not know - a proposition.

Propositions often contain equations, which are statements asserting the equality of two expressions containing variables. When writing proofs (including proofs in Coq), we exercise equational reasoning: we apply axioms to equations in order to incrementally transform them into something that is clearly true.

\subsection{The Coq proof assistant}
Many proofs in mathematics or computer science are natural language proofs - that is, they are written in a natural language, like English. Since natural languages are often ambiguous, natural language proofs are susceptible to misinterpretation or misconception.

On the other hand, just as there programming languages that express a set of instructions to be executed by a computer, there are also domain-specific languages for writing formal proofs that can be automatically, or mechanically, verified by a computer. Coq allows us to write formal, verifiable proofs in a structured logical language called Gallina, and will also automatically verify that our proofs are syntactically correct as well as type-correct. See \cite{coq-homepage}.

\subsection{YSC3236 Functional Programming and Proving (FPP)}
FPP is taken not only by Yale-NUS students, but also PhD and post-doctoral students from the National University of Singapore (NUS) School of Computing (SoC). Through the course, students gain an appreciation for the interconnectedness of computer programs and logical proofs - which have previously been presented to them as distinct domains of knowledge. For example, they are led to realize that a Coq proof exactly corresponds to an equivalent mathematical proof they have written in detail, by hand.

Students engage in weekly assignments consisting of rigorous, progressive exercises involving:

\begin{itemize}
    \item writing mathematical proofs
    \item writing programs, and proofs about the properties of programs
    \item eventually, stating their own theorems and proving them.
\end{itemize}

By the end of the course, students will have independently written more proofs than they have ever written in their lives, all of which would have been formally verified.

\subsection{The GNU Emacs Editor}
Emacs (\cite{emacs-homepage}) is a family of real-time text editors characterized by their customizability and extensibility. GNU Emacs was written in 1984 by GNU Project founder Richard Stallman. At Yale-NUS College, GNU Emacs is used in Intro CS, Intro to Algos and Data Structures, and FPP. GNU Emacs provides a language called Emacs Lisp (\cite{emacs-lisp-homepage}) that is used to write programs run within Emacs. The \code{proof-reader} tool uses Emacs Lisp to provide a user interface.

\subsection{Proof General}
Proof General (\cite{pg-homepage}) is an Emacs interface for proof assistants, developed at the University of Edinburgh since 1992. It provides a common interface across various proof assistants, including Coq, and allows users to interactively edit proof scripts. The \code{proof-reader} tool interacts with Proof General functions in order to provide its functionality.