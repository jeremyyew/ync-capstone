% Chapter Template

\chapter{Solution: The \code{proof-reader} tool} % Main chapter title

\label{solution} % Change X to a consecutive number; for referencing this chapter elsewhere, use \ref{ChapterX}

\section{Parsing student submissions with \code{proof-reader}}
The \code{proof-reader} tool is a parser that emits two types of warnings corresponding to the syntax issues described in the previous section:

\begin{enumerate}
    \item Warns user of instances where unpermitted tactics are used.
    \item Warns user of instances of incorrect arity in terms supplied to tactics such as "rewrite", "exact", "apply", etc.
\end{enumerate}

\code{proof-reader} is intended to be used by students, both during proof editing and as a final check before submission. As students are writing proofs, the \code{proof-reader}  keeps them on the right track by correcting issues that might be affecting their thought process. When used as a final check, it will help them correct proofs that might have been accepted by Coq but do not demonstrate the intended learning goals of the exercise. The \code{proof-reader} can be used directly on student submissions by the instructor as well.


\section{Usage}
To run \code{proof-reader} on your proof script, simply execute the following Emacs command while in Proof General, with the editor focused on the buffer containing the script:
\begin{minted}{text}
M-x jeremy-proof-reader
\end{minted}
The script will first be re-run from the beginning by Proof General. \code{proof-reader} will then evaluate the script and display any relevant warnings in the Emacs response buffer, for example:
\begin{minted}[fontsize=\scriptsize]{text}
WARNING: In tactic invocation REWRITE on parent term (Nat.add_assoc a b):
Term "Nat.add_assoc" with arity 3 incorrectly applied to 2 terms (a),(b).
\end{minted}
Or, if there are no warnings:
\begin{minted}{text}
No warnings.
\end{minted}

\section{Setup}
Simply download the source code or binary package and follow the installation steps from this repository: \url{https://github.com/jeremyyew/ync-capstone}.

\section{Examples}

\subsection{Example 1: Warning user of instances where unpermitted tactics are used}
\label{example-1}
When \code{proof-reader} is applied to the example proof script in the section \ref{abuse-of-tactics} \nameref{abuse-of-tactics}, the output is:
\begin{minted}{text}
Parser error: Could not parse the substring "trivial.". "trivial" may be an unpermitted tactic, please only use tactics that have been introduced in the course.
\end{minted}

\subsection{Example 2: Warning user of instances of incorrect arity}
\label{example-2}
When \code{proof-reader} is applied to the example proof script in the section \ref{misuse-of-tactics} \nameref{misuse-of-tactics}, the output is:

\begin{minted}[fontsize=\scriptsize]{text}
WARNING: In tactic invocation REWRITE on parent term (Nat.add_assoc a b):
Term "Nat.add_assoc" with arity 3 incorrectly applied to 2 terms (a),(b).

WARNING: In tactic invocation REWRITE on parent term (Nat.add_assoc (a+b)):
Term "Nat.add_assoc" with arity 3 incorrectly applied to 1 terms (a+b).

WARNING: In tactic invocation REWRITE on parent term (Nat.add_assoc):
Term "Nat.add_assoc" with arity 3 incorrectly applied to 0 terms.
\end{minted}

\section{Possible errors}
\code{proof-reader}\ only accepts syntactically correct Coq code. It will first trigger Proof General to reevaluate the entire buffer. As long as Proof General accepts the script without error, \code{proof-reader}\ will process it.

If there are Coq syntax errors, \code{proof-reader}\ will display:
\begin{minted}{text}
Coq error raised. Please correct and try again.
\end{minted}
The parser will then terminate without evaluating the script. The Coq errors will be in the response buffer, as usual.

Furthermore, \code{proof-reader}\ only accepts a subset of Coq syntax, which has been pre-defined by the instructor (see "Supported syntax" in the Appendix on the repository). Therefore, if the script contains unsupported syntax, it is likely either a command that has been introduced in the course but not accounted for, or a bug. \code{proof-reader}\ will display:
\begin{minted}{text}
Parser error: Could not parse "XXX". This syntax may not be currently supported.
\end{minted}

To extend the supported syntax or modify the parser behaviour, see \ref{extending} \nameref{extending}.