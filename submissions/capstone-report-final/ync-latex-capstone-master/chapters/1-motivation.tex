% Chapter Template

\chapter{Motivation} % Main chapter title

\label{motivation} % Change X to a consecutive number; for referencing this chapter elsewhere, use \ref{ChapterX}
In this section, we relate some of the learning goals of FPP to a specific problem faced in class, in order to motivate the solution presented. \textbf{This paper's contributions therefore rely on the pedagogical approach taken by Professor Danvy in FPP.} Readers may refer to "Mystery Functions" (\cite{Danvy2020}) – presented as a keynote at the International Symposium on Implementation and Application of Functional Languages 2019 – which conveys the spirit of this approach, its assumptions, and its results.

\section{Building muscle memory}
The learning philosophy of FPP is that programming and proving is similar to training in any skilled discipline such as martial arts, cooking, or dance: beginner training should build muscle memory for basic skills and habits. For example, if you are training to be a chef, but you don’t develop proper knife skills early on, this will hurt you for the rest of your career.

Therefore, in the first half of the course, students complete rigorous, progressive exercises in order to practice specific proof techniques and programming habits. In the second half of the course, students can then rely on this muscle memory to write proofs with greater creativity and efficiency.

\section{Syntax issues}
Just as there are many ways to write the same program, there are many equivalent versions of a Coq proof, especially because Coq is flexible and allows you to take shortcuts. However, for new learners, this power can be counterproductive. In the context of FPP, several issues arise.


\subsection{Abuse of tactics}
\label{abuse-of-tactics}
First, students may abuse tactics that have not been introduced in the course. When students get stuck on a proof, they might Google for related solutions or search the Coq documentation for anything that will solve the proof. They might end up using a 'magical' tactic, for example the tactic ‘trivial‘, as in the example proof below.

\begin{minted}{Coq}
Lemma SSSn_is_3_plus_n :
    forall n : nat,
    S (S (S n)) = 3 + n.
Proof.
    trivial.
Qed.
\end{minted}
Under the hood, the ‘trivial‘ tactic iterates over various strategies to solve the current formula. However, in the first half of the course the focus is for students to understand every single proof step they write. Therefore, using a tactic like ‘trivial' goes against the objective of the exercise. Instead, the proof should demonstrate every step:

\begin{minted}{Coq}
...
Proof.
    intro n.
    rewrite <- (Nat.add_1_l n).
    rewrite <- (plus_Sn_m 1 n).
    rewrite <- (plus_Sn_m 2 n).
    reflexivity.
Qed.
\end{minted}
Yet these tactics still appear in student submissions. This causes time between resubmissions to be wasted on superficial feedback.

\subsection{Misuse of tactics}
\label{misuse-of-tactics}
Second, even when students use tactics that have been introduced, they may misuse them. For instance, the rewrite tactic is used to apply a rewrite rule to the current goal. A rewrite rule is a function that expects arguments that refer to corresponding terms in the goal; Coq will rewrite the corresponding terms in the goal. For example, the rewrite rule \code{Nat.add\_assoc} accepts three arguments, \code{n}, \code{m}, and \code{p}:

\begin{minted}{Coq}
Check Nat.add_assoc.
# Nat.add_assoc : forall n m p : nat, n + (m + p) = n + m + p.
\end{minted}
However, Coq is flexible with the number of arguments given to terms. As the example proof below demonstrates, you could give the rewrite rule three, two, one or zero of the rewrite arguments required, and Coq will simply infer the intended application, by picking the first terms in the formula that it can apply the rule to.

\begin{minted}{Coq}
Proposition add_assoc_nested :
    forall a b c d e: nat,
    a + b + c + d + e = a + (b + (c + (d + e))).
Proof.
    intros a b c d e.
    rewrite -> (Nat.add_assoc a b).
    rewrite -> (Nat.add_assoc (a + b)).
    rewrite -> Nat.add_assoc.
    reflexivity.
Qed.
\end{minted}
However, in FPP, tactic applications should be explicit, i.e. rewrite rules should be supplied with the exact number of arguments required, so that it is clear which part of the goal is being rewritten:

\begin{minted}{Coq}
...
Proof.  
  intros a b c d e.  
  rewrite -> (Nat.add_assoc a b (c + (d + e))).  
  rewrite -> (Nat.add_assoc (a + b) c (d + e)).  
  rewrite -> (Nat.add_assoc (a + b + c) d e).  
  reflexivity.
Qed.
\end{minted}
This issue arises for the tactics \code{rewrite}, \code{exact} and \code{apply}.

\section{The goal: automated intervention on syntax issues}
These two issues - abuse and misuse of tactics - correspond to issues of \emph{abstract syntax} (what language constructs are represented in the grammar) and \emph{concrete syntax} (what structures are used to represent language constructs) respectively. These issues are often highlighted in written feedback from the instructor, resulting in much 'superficial' feedback. Superficial feedback is comments on syntax issues, whereas substantive feedback is comments on the logical content of the proof.

\code{proof-reader} is a tool that anticipates and identifies both abstract and concrete syntax issues. By automatically intervening during the proof editing process, \code{proof-reader} guides students towards solutions that do not require superficial feedback, allowing the instructor to focus on substantive feedback.